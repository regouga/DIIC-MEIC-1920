\documentclass{sigchi-ext}
% Please be sure that you have the dependencies (i.e., additional
% LaTeX packages) to compile this example.
\usepackage[T1]{fontenc}
\usepackage{textcomp}
\usepackage[scaled=.92]{helvet} % for proper fonts
\usepackage{graphicx} % for EPS use the graphics package instead
\usepackage{balance}  % for useful for balancing the last columns
\usepackage{booktabs} % for pretty table rules
\usepackage{ccicons}  % for Creative Commons citation icons
\usepackage{ragged2e} % for tighter hyphenation
\usepackage{cite}

% Para adicionar a língua (PT)
\usepackage[portuges]{babel}
% suporte para utf8
\usepackage[utf8]{inputenc}

% Some optional stuff you might like/need.
% \usepackage{marginnote} 
% \usepackage[shortlabels]{enumitem}
% \usepackage{paralist}
% \usepackage[utf8]{inputenc} % for a UTF8 editor only

%% EXAMPLE BEGIN -- HOW TO OVERRIDE THE DEFAULT COPYRIGHT STRIP --
% \copyrightinfo{Permission to make digital or hard copies of all or
% part of this work for personal or classroom use is granted without
% fee provided that copies are not made or distributed for profit or
% commercial advantage and that copies bear this notice and the full
% citation on the first page. Copyrights for components of this work
% owned by others than ACM must be honored. Abstracting with credit is
% permitted. To copy otherwise, or republish, to post on servers or to
% redistribute to lists, requires prior specific permission and/or a
% fee. Request permissions from permissions@acm.org.\\
% {\emph{CHI'14}}, April 26--May 1, 2014, Toronto, Canada. \\
% Copyright \copyright~2014 ACM ISBN/14/04...\$15.00. \\
% DOI string from ACM form confirmation}
%% EXAMPLE END

% Paper metadata (use plain text, for PDF inclusion and later
% re-using, if desired).  Use \emtpyauthor when submitting for review
% so you remain anonymous.
\def\plaintitle{Eco Footprint - Understanding my impact} \def\plainauthor{João Nogueira, João Pina, Manuel Sousa,
  Miguel Regouga}
\def\emptyauthor{}
\def\plainkeywords{Authors' choice; of terms; separated; by
  semicolons; include commas, within terms only; required.}
\def\plaingeneralterms{Documentation, Standardization}

\title{Eco Footprint - Understanding my impact}

\numberofauthors{6}
% Notice how author names are alternately typesetted to appear ordered
% in 2-column format; i.e., the first 4 autors on the first column and
% the other 4 auhors on the second column. Actually, it's up to you to
% strictly adhere to this author notation.
\author{%
  \alignauthor{%
    \textbf{João Nogueira}\\
    \affaddr{Student number: 83488} \\
    \affaddr{Instituto Superior Técnico} \\
    \email{jrnogueira@edu.ulisboa.pt} }\alignauthor{%
    \textbf{João Pina}\\
    \affaddr{Student number: 85080}\\
    \affaddr{Instituto Superior Técnico}\\
    \email{joaomfpina@tecnico.ulisboa.pt} } \vfil \alignauthor{%
    \textbf{Manuel Sousa}\\
    \affaddr{Student number: xxxxx}\\
    \affaddr{Instituto Superior Técnico}\\
    \\
    \email{mail@mail.com} }\alignauthor{%
    \textbf{Miguel Regouga}\\
    \affaddr{Student number: 83530}\\
    \affaddr{Instituto Superior Técnico}\\
    \email{miguelregouga@tecnico.ulisboa.pt} } \vfil}

% Make sure hyperref comes last of your loaded packages, to give it a
% fighting chance of not being over-written, since its job is to
% redefine many LaTeX commands.
\definecolor{linkColor}{RGB}{6,125,233}
\hypersetup{%
  pdftitle={\plaintitle},
%  pdfauthor={\plainauthor},
  pdfauthor={\emptyauthor},
  pdfkeywords={\plainkeywords},
  bookmarksnumbered,
  pdfstartview={FitH},
  colorlinks,
  citecolor=black,
  filecolor=black,
  linkcolor=black,
  urlcolor=linkColor,
  breaklinks=true,
}

% \reversemarginpar%

\begin{document}

%% For the camera ready, use the commands provided by the ACM in the Permission Release Form.
\CopyrightYear{2019}
\setcopyright{rightsretained}
\conferenceinfo{CHI'20,}{April  25--30, 2020, Honolulu, HI, USA}
\isbn{978-1-4503-6819-3/20/04}
\doi{https://doi.org/10.1145/3334480.XXXXXXX}
%% Then override the default copyright message with the \acmcopyright command.
\copyrightinfo{\acmcopyright}


\maketitle \RaggedRight{} 


\keywords{\plainkeywords}


\section{Introduction}
Nowadays, the environment is one of the most talked topics in the entire world. Sea levels are rising, there's tons of plastic in the ocean, the world is getting warmer, levels of CO2 emissions are increasing every day, among many other problems that ultimately lead to a degrading state of the world. Today's society still ignores these problems that won't be paid in their generation, and green habits are only respected by unrepresented small groups of people that try to keep the longevity of this planet. We believe people can do better without radical changes in their lives, if they make the small effort of paying attention to some of their behaviours and actions that are done in a daily basis. If people could, for instance, figure out the amount of water wasted during their morning shower, their usage of unnecessary lightning, or even the food that is wasted and put on the trash, people would be more aware of how their ecological behaviour and change it for better. Our ultimate goal is to make use of technology to bring people to the attention that they can do better, not only to the environment but also to their wallets.


\section{Proposed solution}
\subsection{Approach}

\subsection{It's all about the garbage can}
To calculate (and, consequently, improve) the ecological footprint of a user, many objects or activities could be tracked. Water, electricity, gas, natural gas, driving, waste, just to name a few. For our project, our group decided to focus on the garbage that is produced in households. We believe that the waste that we produce is one crucial point of world destabilisation; starting with the plastic that we consume leads up in the oceans, the food we waste that could be feeding someone who has less economic resources, the general rubbish that ends up in polluted landfills - all of this starts in our garbage can.

\subsection{A case study: the ambient orb}
A great and useful design is strictly necessary in a project that aims to change users' behaviour - in our case, to reduce their ecological footprint.

The ambient orb is...

\subsection{Self-awareness}
Given how easy it is for the user to understand the ambient orb, our solution is based on the ambient orb. A simple LED colored light, placed in a strategic position (such as the kitchen) is more than enough for a person to understand how good or bad is their ecological footprint. The light would be accordingly adjusted: red if a user is doing little for the environment (in our case, if the amount of garbage produced is above the average), green if the user is fully committed (if the amount of waste is below average), or yellow (if it's somewhere in between). This would lead to 

\subsection{Peer pressure}
This idea, however, generates an incon

\subsection{Materials}
In order to put in practice and assemble our idea, we aim in the first place to build a functional physical prototype, using the Arduino board. The following materials will be required:

\begin{itemize}
	\item \textbf{Garbage can (3 to 4 units)} - each garbage can has built-in sensors that determine the amount of volume present; these values are then sent through Wi-Fi
	\begin{itemize}
		\item 2x AA batteries
		\item 1x Battery holder case
		\item 1x HC - SR04 Ultrasonic Distance Sensor
		\item 1x Jumper Wires
		\item 1x Arduino Board
		\item 1x Feather HUZZAH w/ ESP8266 WiFi
		\item 1x Plastic 50l garbage can
		\item 1x Coloured garbage can cover (grey for common trash; green for glass; yellow for plastic and metal; blue for paper and card)
	\end{itemize}
	\item \textbf{Light bulb (indoor use)} - the light bulb acts as an 'ambient orb', and it is placed in a convenient place to the user to grant self-awareness
	\begin{itemize}
		\item 1x Xiaomi Mi LED Smart Bulb Yeelight Wi-Fi
		\item 1x Small lamp - IKEA's Fado
	\end{itemize}
	\item \textbf{Portable LED (outdoor use)} - when not in home, users can track their eco footprint on the go with a portable LED that can be attached to a keychain or the back of a smartphone
	\begin{itemize}
		\item 1x Adafruit Mini Skinny NeoPixel Digital RGB LED Strip - 60 LED/m (1m)
		\item 1x Lithium Ion Polymer Battery - 3.7v 150mAh
		\item 1x Adafruit nRF52840 Feather or equivalent
		\item 1x Breadboard-friendly SPDT Slide Switch
		\item 1x JST-PH Battery Extension Cable - 500mm
	\end{itemize}
\end{itemize}

On the other side, we also want to build a simple prototype application that should not only allow the interaction between the user and the system, but also to give the user an in-depth analysis of their ecological footstep, so they can act upon their behaviours. Thus, for the software part, only a computer and a smartphone running Android or iOS are necessary.


\section{Introduction}


\begin{table}
  \centering
  \begin{tabular}{l r r r}
    % \toprule
    & & \multicolumn{2}{c}{\small{\textbf{Test Conditions}}} \\
    \cmidrule(r){3-4}
    {\small\textit{Name}}
    & {\small \textit{First}}
      & {\small \textit{Second}}
    & {\small \textit{Final}} \\
    \midrule
    Marsden & 223.0 & 44 & 432,321 \\
    Nass & 22.2 & 16 & 234,333 \\
    Borriello & 22.9 & 11 & 93,123 \\
    Karat & 34.9 & 2200 & 103,322 \\
    % \bottomrule
  \end{tabular}
  \caption{Table captions should be placed below the table. We
    recommend table lines be 1 point, 25\% black. Minimize use of
    table grid lines.}~\label{tab:table1}
\end{table}

\begin{itemize}\compresslist%
\item Write in a straightforward style. Use simple sentence
  structure. Try to avoid long sentences and complex sentence
  structures. Use semicolons carefully.
\item Use common and basic vocabulary (e.g., use the word ``unusual''
  rather than the word ``arcane'').
\item Briefly define or explain all technical terms. The terminology
  common to your practice/discipline may be different in other design
  practices/disciplines.
\item Spell out all acronyms the first time they are used in your
  text. For example, ``World Wide Web (WWW)''.
\item Explain local references (e.g., not everyone knows all city
  names in a particular country).
\item Explain ``insider'' comments. Ensure that your whole audience
  understands any reference whose meaning you do not describe (e.g.,
  do not assume that everyone has used a Macintosh or a particular
  application).
\item Explain colloquial language and puns. Understanding phrases like
  ``red herring'' requires a cultural knowledge of English. Humor and
  irony are difficult to translate.
\item Use unambiguous forms for culturally localized concepts, such as
  times, dates, currencies, and numbers (e.g., ``1-5- 97'' or
  ``5/1/97'' may mean 5 January or 1 May, and ``seven o'clock'' may
  mean 7:00 am or 19:00). For currencies, indicate equivalences:
  ``Participants were paid {\fontfamily{txr}\selectfont \textwon}
  25,000, or roughly US \$22.''
\item Be careful with the use of gender-specific pronouns (he, she)
  and other gender-specific words (chairman, manpower,
  man-months). Use inclusive language (e.g., she or he, they, chair,
  staff, staff-hours, person-years) that is gender-neutral. If
  necessary, you may be able to use ``he'' and ``she'' in alternating
  sentences, so that the two genders occur equally
  often~\cite{Schwartz:1995:GBF}.
\item If possible, use the full (extended) alphabetic character set
  for names of persons, institutions, and places (e.g.,
  Gr{\o}nb{\ae}k, Lafreni\'ere, S\'anchez, Nguy{\~{\^{e}}}n,
  Universit{\"a}t, Wei{\ss}enbach, Z{\"u}llighoven, \r{A}rhus, etc.).
  These characters are already included in most versions and variants
  of Times, Helvetica, and Arial fonts.
\end{itemize}

% \begin{figure}
%   \includegraphics[width=.9\columnwidth]{figures/ea-figure2}
%   \caption{If your figure has a light background, you can set its
%     outline to light gray, like this, to make a box around
%     it.}\label{fig:bats}
% \end{figure}

\begin{marginfigure}[-35pc]
  \begin{minipage}{\marginparwidth}
    \centering
    \includegraphics[width=0.9\marginparwidth]{figures/cats}
    \caption{In this image, the cats are tessellated within a square
      frame. Images should also have captions and be within the
      boundaries of the sidebar on page~\pageref{sec:sidebar}. Photo:
      \cczero~jofish on Flickr.}~\label{fig:marginfig}
  \end{minipage}
\end{marginfigure}




\balance{} 

\bibliographystyle{SIGCHI-Reference-Format}
\bibliography{sample}

\end{document}

%%% Local Variables:
%%% mode: latex
%%% TeX-master: t
%%% End:
